\section{Introduction}
\label{sec:intro}

%\begin{enumerate}
%\item Need for efficient Surrogates - especially for complex models to enable forward UQ, 
%sensitivity studies, calibration, and experimental design (Include citations).
%\item Computational Hurdles: Polynomial Chaos and GP can be computationally 
%intractable and suffer from the curse of dimensionality (Include plots for PCE
%to motivate dimension reduction with citations for both PCE and GP).
%\item Sensitivity analyis - a potent tool for dimension reduction. However, SA can be
%computationally prohibitive. In fact, there are studies demonstrating the use of 
%surrogates to reduce costs associated with SA (include citations). 
%\item DGSM - brief introduction and citations. 
%\item Key contributions of the paper: 1) Methodology that exploits DGSM to reduce the
%dimensionality of the problem and thus enables efficient construction of surrogates.
%2) Application of the proposed methodology to investigate relative importance of
%parameters in the stillinger-weber potential, commonly used for studying phonon transport in silicon. Further, construct a reasonably accurate PC surrogate in the reduced 
%space and demonstrate computational advantage of this approach. .
%\end{enumerate}

