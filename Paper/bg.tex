\section{Background}
\label{sec:bg}

In this section, we provide a brief theoretical background to the
derivative-based global sensitivity approach and the polynomial chaos
expansion.

\subsection{Derivative-based global sensitivity}  

The derivative-based global sensitivity analysis (DGSA) is performed by 
estimating a sensitivity measure~\cite{Kucherenko:2009} for each stochastic
parameter in a model or a function. The
sensitivity measure associated with a parameter, $\theta_i$, based on
the observable, $\mathcal{Y}$ is given as follows:

\be
\mu_i = \mathbb{E}\left[\left(\frac{\partial \mathcal{Y}(\bm{x})}{\partial \theta_i}\right)^{2}\right]
\label{eq:mu}
\ee

\noindent The derivative in the above equation is estimated using finite
difference as given below, and averaged using pseudo-random samples in
the uncertain parameter domain. 

\be
\frac{\partial \mathcal{Y}(\bm{\theta}^{\ast})}{\partial \theta_i} =
 \lim_{\Delta\theta_i^{*}\to 0}
\frac{[\mathcal{Y}(\theta_1^{*},\ldots,\theta_{i-1}^{*},
\theta_i^{*}+\Delta\theta_i^{*},
\theta_{i+1}^{*},\ldots,\theta_d^{*}) - 
\mathcal{Y}(\bm{\theta}^{*})]}{\Delta\theta_i^{*}} 
\ee

\noindent The total number of model realizations or function evaluations
needed to
compute $\mu_i$ in a Euclidian space $\mathbb{R}^d$ using $N$ samples is
therefore, $N\times(d+1)$.

