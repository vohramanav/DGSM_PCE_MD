\section{Background}
\label{sec:bg}

In this section, we provide a brief theoretical background to the
derivative-based global sensitivity measure and the polynomial chaos
expansion.

\subsection{Derivative-based global sensitivity}  

The derivative-based global sensitivity analysis (DGSA) is performed by 
estimating a sensitivity measure~\cite{Sobol:2009} for each stochastic
parameter in a model or a function. The
sensitivity measure associated with a parameter, $\theta_i$, based on
the observable, $\mathcal{Y}$ is given as follows:

\be
\mu_i = \mathbb{E}\left[\left(\frac{\partial \mathcal{Y}(\bm{x})}{\partial \theta_i}\right)^{2}\right]
\label{eq:mu}
\ee

\noindent The derivative in the above equation is estimated using finite
difference as given below, and averaged using pseudo-random samples in
the uncertain parameter domain. 

\be
\frac{\partial \mathcal{Y}(\bm{\theta}^{\ast})}{\partial \theta_i} =
 \lim_{\Delta\theta_i^{*}\to 0}
\frac{[\mathcal{Y}(\theta_1^{*},\ldots,\theta_{i-1}^{*},
\theta_i^{*}+\Delta\theta_i^{*},
\theta_{i+1}^{*},\ldots,\theta_d^{*}) - 
\mathcal{Y}(\bm{\theta}^{*})]}{\Delta\theta_i^{*}} 
\label{eq:partial}
\ee

The total number of model realizations or function evaluations
needed to
compute $\mu_i$ in a Euclidian space $\mathbb{R}^d$ using $N$ samples is
therefore, $N\times(d+1)$. The derivative in Eq.~\ref{eq:partial} could be evaluated
analytically in case the functional dependence of $\mathcal{Y}$ and $\theta_i$
is known. Note that the perturbation $\Delta\theta_i^{*}$ is extremely small 
compared to the uncertainty in $\theta_i$ which leads to a significant improvement
over the `elementary effect' of a parameter, estimated in the Morris method~\cite{Morris:1991}
using large increments. 

For parameters with continuous distributions, the upper bound on the total 
Sobol' sensitivity index ($\mathcal{T}(\theta_i)$)
can be expressed in terms of $\mu_i$, the Poincar\'e constant ($\mathcal{C}_i$) and the total 
variance of the model output ($V$)~\cite{Lamboni:2013}:

\be
\mathcal{T}(\theta_i) \leq \frac{\mathcal{C}_i\mu_i}{V}~(\propto \hat{\mathcal{C}_i\mu_i})
\label{eq:bound}
\ee

\noindent The upper bound in the above inequality is proportional to the product of $\mathcal{C}_i$
and $\mu_i$. However, for the purpose of parameter screening as discussed later in
Section~\ref{sec:method}, we consider a normalized product, $\hat{\mathcal{C}_i\mu_i}$:

\be
\hat{\mathcal{C}_i\mu_i} = \frac{\mathcal{C}_i\mu_i}{\sum_i \mathcal{C}_i\mu_i}
\label{eq:cmu}
\ee

\noindent The Poincar\'e constant, $\mathcal{C}_i$ is specific to the probability distribution of $\theta_i$.
Table~\ref{tab:poincare} provides its value in the case of uniform and normal
distributions.
\bigskip

\begin{table}[htbp]
\renewcommand*{\arraystretch}{1.2}
\begin{center}
\begin{tabular}{|c|c|}
\hline
Distribution & $\mathcal{C}_i$ \\ \hline \hline 
Uniform, $\mathcal{U}[a, b]$ & $(b-a)^{2}/\pi^2$ \\ 
Normal, $\mathcal{N}(\mu,\sigma^2)$ & $\sigma^2$ \\ 
\hline
\end{tabular}
\end{center}

\caption{Poincare constant for the case of uniformly and normally distributed random
parameters~\cite{Roustant:2014}.}
\label{tab:poincare}
\end{table}

\subsection{Polynomial Chaos Expansion}

As discussed earlier in Section~\ref{sec:intro}, the Polynomial Chaos Expansion (PCE)
is one of the most commonly used surrogates for representing the dependence of a random
observable ($\mathcal{Y}$) on independent uncertain or stochastic model parameters ($\bm{\theta}$).
Considering a joint probability distribution of the components of ($\bm{\theta}$) as $\mathbb{P}(\bm{\theta})$,
the following condition is imposed on $\mathcal{Y}$:

\be
\mathbb{E}[\mathcal{Y}^2] = \int_{\mathcal{D}_{\bm{\theta}}} \mathcal{Y}^2 \mathbb{P}(\bm{\theta}) 
d\bm{\theta} < \infty
\ee

\noindent where $\mathcal{D}_{\bm{\theta}}$ is the domain of the input parameter space. The PCE of
$\mathcal{Y}$ ($\mathcal{Y}^{\mbox{\tiny {PC}}}$) is a mean-square 
convergent truncated summation given as
follows~\cite{Xiu:2002,Ghanem:2003,Olivier:2010}:

\be
\mathcal{Y}^{\mbox{\tiny M}} \approx \mathcal{Y}^{\mbox{\tiny PC}} = 
\sum_{\alpha\in\mathcal{I}} c_{\alpha}\Psi_{\alpha}(\bm{\xi(\theta)}) 
\ee

\noindent where $\mathcal{Y}^{\mbox{\tiny M}}$ corresponds to realizations of the
observable using the original (typically physics-based) model. 
Note that the set of parameters, $\bm{\theta}$ are parameterized in terms of canonical random
variables, $\bm{\xi}$ in the above expansion. $\Psi_{\alpha}$'s are multivariate polynomial
basis, orthonormal with respect to the joint probability distribution of $\bm{\xi}$, $\mathcal{I}$
is the multi-index set which identifies the degree of individual polynomials in each term in the
multivariate basis.

Computational strategies available for estimating the PC coefficients ($c_\alpha$'s) typically involve
techniques based on projection or regression. Projection-based methods rely on numerical
quadrature for estimating the following expectation:

\be
\bm{c} = \mathbb{E}[\Psi_\alpha(\bm{\xi})\cdot\mathcal{Y}^{\mbox{\tiny{M}}}]
\approx
\sum_{i=1}^{N} \mathcal{Y}^{\mbox{\tiny{M}}}(\bm{\theta}^{(i)})w^{(i)}\Psi_\alpha(\bm{\xi}^{(i)})
\ee

\noindent where $w^{(i)}$ denotes the weight associated with the quadrature node $i$. 
Regression-based methods such as least angle regression (LAR)~\cite{Efron:2004} and least absolute shrinkage
and selection operator (LASSO)~\cite{Tibshirani:1996} aim to construct a sparse PCE~\cite{Blatman:2008}
by solving a regularized optimization problem:

\be
\hat{\bm{c}} = \mbox{argmin}~\mathbb{E}\left[\left(\bm{c}^{T}\Psi(\bm{\bm{\xi(\theta)}}) -
\mathcal{Y}^{\mbox{\tiny{M}}}(\bm{\theta})\right)^{2}\right]  + \lambda\normone{\bm{c}}
\ee

\noindent where $\normone{\bm{c}}$ = $\sum_{\alpha\in \mathcal{I}} |c_\alpha |$.
The regularization term forces the minimization towards low-rank solutions.
In this work, we construct sparse PCEs with LAR using UQLab~\cite{Marelli:2014},
a general purpose uncertainty quantification software developed at ETH Zurich in Switzerland.


















