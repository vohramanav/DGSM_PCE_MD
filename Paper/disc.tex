\section{Summary and Conclusion}
\label{sec:disc}

%\begin{enumerate}
%\item Methodology is agnostic to the choice of surrogate.
%\item Outcome of parameter screening is QoI dependent. 
%\item As required for a PCE, the uncertain parameters need not be independent. 
% When working with a computational budget, consider relaxing the tolerance. 
% Higher the dimension reduction, greater the gains. However, in case of intensive models, even a small reduction could be
%advantageous. 
% In general, computational advantage is not guaranteed. Nevertheless, the proposed methodology is promising. 
% Gains with a reduced order surrogate are multiplied several time when evaluating the posterior distribution of the parameters
%since several surrogates are needed. 
%\end{enumerate}

In this work, we have presented an efficient and practical approach for constructing
a reduced-space
surrogate for scientific and engineering applications. Dimension reduction is accomplished
by identifying uncertain parameters that contribute relatively less towards the uncertainty
in the quantity of interest. These parameters deemed as \textit{unimportant} are determined
using a screening metric~\eqref{eq:cmu} involving derivative-based sensitivity
measures. Initially, the metric is estimated
using model evaluations at a small set of samples in the parameter domain. However, these
estimates are refined by subsequent enrichment of the sample set during the screening
procedure presented in Algorithm~\ref{alg:screen}. The outcome of parameter screening is
assessed for the scope of dimension reduction. In a favorable scenario, a reduced-space
surrogate (RSS) is constructed. The RSS is tested for accuracy in a least-squares sense
as well as a probabilistic sense using a cross-validation test suite. In the proposed framework,
a surrogate in the full-space (FSS) is constructed in tandem with parameter screening using
the available set of model evaluations. Both, RSS and FSS are
constructed using regression-based sparse PCEs. Simultaneous construction
of the FSS ensures that the computational effort associated with the proposed framework
does not overshoot the effort required to construct the FSS directly.
Hence, the RSS is constructed only in situations where computational gains are
expected. 

Parameter screening methodology was implemented to low-to-moderate dimensional
test problems and an accurate RSS was constructed to demonstrate potential for
computational gains in each case. Furthermore, the overall framework was
implemented  to a relatively higher dimensional application involving kinetics
of the H$_2$/O$_2$ reaction mechanism.  Significant dimension reduction (19 
dimensions to 3 or 4 dimensions) was
accomplished for two different scenarios involving a fuel-rich and a fuel-lean
mixture. In both cases, the resulting RSS was able to capture the input-output
relationship as well as the uncertainty in the quantity of interest with
reasonable accuracy. Additional highlights of the proposed framework are as
follows:
\begin{enumerate}
\item Although PCEs were used in this work, the proposed framework is
agnostic to the choice of the surrogate model construction method.
%\item The quantity of interest must be differentiable with respect to each uncertain parameter
%for the derivative-based sensitivity measures in~\eqref{eq:mu} to be defined. 
%to ensure accurate estimation of the derivative-based sensitivity measures in~\eqref{eq:mu}.
\item Substantial computational gains are expected in situations involving
compute-intensive simulations even if the scope for dimension reduction is
small. 
%Hence, careful judgment pertaining to the construction of a reduced-order
%surrogate as well as resource allocation is needed. is required when
%implementing the framework. 

\item Exponential gains can be realized in situations where multiple surrogates need to be
constructed as illustrated in the kinetics application. Other possible scenarios may
include inverse problems involving parameter estimation in a Bayesian setting.

\item Dimension reduction based on the proposed methodology could help reduce the
effort required for model calibration wherein only the important parameters
are calibrated. 
\end{enumerate}

Based on the results presented for the test problems and the kinetics
application, the proposed framework seems quite promising in its
potential for identifying the unimportant model inputs. These
observations could be exploited to construct efficient model surrogates
in a reduced input space. However, it is important to remain cognizant
about the limitations of the framework as well. 
For instance, the quantity of interest is required to be differentiable with
respect to each parameter in the considered domain. This condition once
satisfied, helps improve the DGSM estimates and enhances the accuracy
of the PCE-based surrogates as well.  
Additionally, the proposed framework does not
account for the existence of possible correlations between the uncertain
inputs of the model. However, while the assumption of independent
inputs is not always justified, in many cases, correlations between
inputs are not well understood a priori, and assuming mutual independence
could be reasonable at least in initial screening using DGSMs. On the other
hand, if approximate correlations are known, we recommend using a Gaussian 
process or Kriging-based surrogate since it 
provides a means for incorporating the correlation between inputs.  
Implementation to applications involving strongly correlated parameters
could enhance the applicability of the proposed framework. 
We consider that to be a potential direction for future studies related
to this work. 
















