\section{Summary and Discussion}
\label{sec:disc}

%\begin{enumerate}
%\item Methodology is agnostic to the choice of surrogate.
%\item Outcome of parameter screening is QoI dependent. 
%\item As required for a PCE, the uncertain parameters need not be independent. 
% When working with a computational budget, consider relaxing the tolerance. 
% Higher the dimension reduction, greater the gains. However, in case of intensive models, even a small reduction could be
%advantageous. 
% In general, computational advantage is not guaranteed. Nevertheless, the proposed methodology is promising. 
% Gains with a reduced order surrogate are multiplied several time when evaluating the posterior distribution of the parameters
%since several surrogates are needed. 
%\end{enumerate}

Constructing model surrogates in a large-dimensional setting can be computationally
challenging especially if the simulations are compute-intensive. In this study, we
have presented a framework that aims to exploit parametric sensitivity to identify
\textit{unimportant} parameters in the model. The uncertainty associated with the
parameters deemed unimportant is disregarded, and the surrogate is constructed in a
low-dimensional subspace. Conventional approaches on the contrary, use a surrogate
to perform global sensitivity analysis in order to reduce computational costs. However, 
constructing a reliable surrogate in high dimensions could still impose a great
deal of computational burden. The proposed framework strives to tackle this challenge by
estimating the derivative-based global sensitivity measures, shown to converge to the
upper bound on Sobol total-effect index with fewer computations.

A screening metric is defined in~\eqref{eq:cmu} for a quantitative assessment of 
parameter importance. The metric is estimated for a small set of model evaluations
initially. Subsequent estimates are obtained as the number of evaluations increase
with each iteration of the screening procedure in Algorithm~\ref{alg:screen}. 
Once the stopping criterion is met, the metric values are assessed for the
possibility of dimension reduction. In a favorable scenario of possible dimension
reduction, a reduced-order surrogate (ROS) is constructed and modified for accuracy. 
Otherwise, the inputs to the screening procedure are updated as needed and the
process is repeated. Concurrently, a surrogate in the full space (FSS) is constructed
each time a new set of model evaluations is available. Both, ROS and FSS are
constructed using regression-based sparse basis methods. Simultaneous construction
of the FSS ensures that computational effort associated with the proposed framework
does not overshoot the effort required to construct a converged FSS directly.
Hence, the ROS is only constructed in situations where computational gains are
expected. 

The proposed framework was implemented to low-to-moderate dimensional test problems
and a reliable ROS was constructed in each case. Additionally, potential for large
computational gains were demonstrated for a relatively higher dimensional application
involving kinetics of the H$_2$/O$_2$ reaction mechanism. Although PC surrogates were
used in each case, the frameowork is agnostic to the choice of the surrogate. The scope
for dimension-reduction and associated computational gains are dependent on the
application as well as the choice of the quantity of interest that must be differentiable
with respect to each parameter. Substantial gains are possible in situations 
involving intensive computations even if the scope for dimension reduction is small. 
Moreover, exponential gains can be realized with the proposed framework when multiple
surrogates are to be constructed for investigating the uncertainty in the QoI as 
illustrated in the kinetics application or inverse problems involving parameter
estimation in a Bayesian setting.  
