\section{Summary and Conclusion}
\label{sec:disc}

%\begin{enumerate}
%\item Methodology is agnostic to the choice of surrogate.
%\item Outcome of parameter screening is QoI dependent. 
%\item As required for a PCE, the uncertain parameters need not be independent. 
% When working with a computational budget, consider relaxing the tolerance. 
% Higher the dimension reduction, greater the gains. However, in case of intensive models, even a small reduction could be
%advantageous. 
% In general, computational advantage is not guaranteed. Nevertheless, the proposed methodology is promising. 
% Gains with a reduced order surrogate are multiplied several time when evaluating the posterior distribution of the parameters
%since several surrogates are needed. 
%\end{enumerate}

In this work, we have presented a systematic approach for constructing a reduced-order
surrogate for scientific and engineering applications. dimension reduction is accomplished
by identifying uncertain parameters that contribute relatively less towards the uncertainty
in the quantity of interest. These parameters deemed as \textit{unimportant} are determined
using a screening metric, defined in~\eqref{eq:cmu}. Initially, the metric is estimated
using model evaluations at a small set of samples in the parameter domain. However, these
estimates are refined by subsequent enrichment of the sample set during the screening
procedure presented in Algorithm~\ref{alg:screen}. The outcome of parameter screening is
assessed for the scope of dimension reduction. In a favourable scenario, a reduced-order
surrogate (ROS) is constructed. The ROS is tested for accuracy in a least-squares sense
as well as a probabilistic sense using a validation test suite. In the proposed framework,
a surrogate in the full-space (FSS) is constructed in tandem with parameter screening using
the available set of model evaluations. Both, ROS and FSS are
constructed using regression-based sparse basis methods. Simultaneous construction
of the FSS ensures that the computational effort associated with the proposed framework
does not overshoot the effort required to construct the FSS directly.
Hence, the ROS is constructed only in situations where computational gains are
expected. 

Parameter screening methodology was implemented to low-to-moderate dimensional test problems
and an accurate ROS was constructed to demonstrate potential for computational gains in
each case. Furthermore, the overall framework was implemented  to a relatively higher 
dimensional application involving kinetics of the H$_2$/O$_2$ reaction mechanism. 
Significant dimension reduction was accomplished for two different scenarios involving
a fuel-rich and a fuel-lean mixture. In both cases, the resulting ROS was able to capture
the input-output relationship as well as the uncertainty in the quantity of interest with 
reasonable accuracy. Additional highlights of the proposed framework are as follows:

\begin{enumerate}
\item Although Polynomial Chaos surrogates were used in this work, the framework is
agnostic to the choice of the surrogate. 
\item The scope for dimension reduction and associated computational gains are
dependent upon the application as well as the quantity of interest. 
\item The quantity of interest must be differentiable with respect to each uncertain parameter
to ensure accurate estimation of the derivative-based sensitivity measures in~\eqref{eq:mu}.
\item Substantial computational gains are expected in situations involving compute-intensive
simulations even if the scope for dimension reduction is small. Hence, careful judgment 
is required when implementing the framework. 
\item Exponential gains can be realized in situations where multiple surrogates need to be
constructed as illustrated in the kinetics application. Other possible scenarios include
include inverse problems involving parameter estimation in a Bayesian setting. 
\end{enumerate}

%Although PC surrogates were
%used in each case, the frameowork is agnostic to the choice of the surrogate. The scope
%for dimension-reduction and associated computational gains are dependent on the
%application as well as the choice of the quantity of interest that must be differentiable
%with respect to each parameter. Substantial gains are possible in situations 
%involving intensive computations even if the scope for dimension reduction is small. 
%Moreover, exponential gains can be realized with the proposed framework when multiple
%surrogates are to be constructed for investigating the uncertainty in the QoI as 
%illustrated in the kinetics application or inverse problems involving parameter
%estimation in a Bayesian setting.  
