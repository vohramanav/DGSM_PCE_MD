\section{Application: H$_2$/O$_2$ Reaction Kinetics}
\label{sec:app}

%Problem set-up: something about the reaction - why is it important?
%different reactions, reaction rate definition, uncertain parameters,
%quantity of interest. 
%
%Implementation of the proposed methodology for two scenarios: lean
%mix and rich mix. Describe what we mean by lean and rich using
%stoichiometry. 
%
%Define all the inputs as per the framework and illustrate its
%implementation to this application for both scenarios

The proposed framework in section~\ref{sec:method} is implemented to 
the H$_2$/O$_2$ reaction mechanism provided in~\cite{Yetter:1991}.
The mechanism comprises of 19 reactions including chain reactions,
dissociation/recombination reactions, and formation and consumption
of intermediate species: HO$_2$ and H$_2$O$_2$, as provided below
in Table~\ref{tab:kinetics}.

\begin{table}[htbp]
\renewcommand*{\arraystretch}{1.2}
\begin{center}
\begin{tabular}{ll}
\toprule
$\mathcal{R}_1$ & H + O$_2$ $\Longleftrightarrow$ O + OH \\
$\mathcal{R}_2$ & O + H$_2$ $\Longleftrightarrow$ H + OH \\
$\mathcal{R}_3$ & H$_2$ + OH $\Longleftrightarrow$ H$_2$O + H \\
$\mathcal{R}_4$ & OH + OH $\Longleftrightarrow$ O + H$_2$O \\
$\mathcal{R}_5$ & H$_2$ + M $\Longleftrightarrow$ H + H + M \\
$\mathcal{R}_6$ & O + O + M $\Longleftrightarrow$ O$_2$ + M \\
$\mathcal{R}_7$ & O + H + M $\Longleftrightarrow$ OH + M \\
$\mathcal{R}_8$ & H + OH +M $\Longleftrightarrow$ H$_2$O + M \\
$\mathcal{R}_9$ & H + O$_2$ + M $\Longleftrightarrow$ HO$_2$ + M \\
$\mathcal{R}_{10}$ & HO$_2$ + H $\Longleftrightarrow$ H$_2$ + O$_2$ \\
$\mathcal{R}_{11}$ & HO$_2$ + H $\Longleftrightarrow$ OH + OH \\
$\mathcal{R}_{12}$ & HO$_2$ + O $\Longleftrightarrow$ O$_2$ + OH \\
$\mathcal{R}_{13}$ & HO$_2$ + OH $\Longleftrightarrow$ H$_2$O + O$_2$ \\
$\mathcal{R}_{14}$ & HO$_2$ + HO$_2$ $\Longleftrightarrow$ H$_2$O$_2$ + O$_2$ \\
$\mathcal{R}_{15}$ & H$_2$O$_2$ + M $\Longleftrightarrow$ OH + OH + M \\
$\mathcal{R}_{16}$ & H$_2$O$_2$ + H $\Longleftrightarrow$ H$_2$O + OH \\
$\mathcal{R}_{17}$ & H$_2$O$_2$ + H $\Longleftrightarrow$ HO$_2$ + H$_2$ \\
$\mathcal{R}_{18}$ & H$_2$O$_2$ + O $\Longleftrightarrow$ OH + HO$_2$ \\
$\mathcal{R}_{19}$ & H$_2$O$_2$ + OH $\Longleftrightarrow$ HO$_2$ + H$_2$O \\
\bottomrule
\end{tabular}
\end{center}

\caption{Reaction mechanism for H$_2$/O$_2$~\cite{Yetter:1991}.}
\label{tab:kinetics}
\end{table}

The reaction rate for the $i^{th}$ reaction as a function of temperature
is given as follows:

\be
k_i(T) = A_iT^{n_i}\exp(-E_{a,i}/RT) 
\label{eq:rate}
\ee
%
where $A_i$ is the pre-exponent, $n_i$ is the index of $T$, $E_{a,i}$
is the activation energy corresponding to the $i^{th}$ reaction, and
$R$ is the universal gas constant. We focus our attention on 
quantifying the uncertainty in the `ignition delay' due to uncertainty
associated with the pre-exponent, $A_i$ for each reaction. Hence, the
total number of uncertain parameters in the present case is 19. 
While the dimensionality of the problem is relatively moderate,
constructing a surrogate in the 19D parameter space could still be
expensive. Hence, we explore the possibility of constructing a
reduce-order surrogate (ROS) using the framework presented earlier
in section~\ref{sec:method}. To this end, the $A_i$'s are considered
to be uniformly distributed in the interval: $[0.9A_i^\ast, 1.1A_i^\ast]$;
$A_i^\ast$ being the nominal estimate corresponding to the $i^{th}$
reaction. TChem~\cite{Safta:2011}, an open source C package was used
for simulating temperature and concentration evolution during the
progress of reactions ($\mathcal{R}_1$ -- $\mathcal{R}_{19}$), and determining
the ignition delay. The set of nominal values used in the computations,
for parameters in~\eqref{eq:rate} are provided in~\cite{Yetter:1991}. 

In this study, we focus on two scenarios: fuel(H$_2$) rich, and fuel(H$_2$)
lean. Consider the global reaction:
%
\be
2\text{H}_2 + \text{O}_2 \rightarrow 2\text{H}_2\text{O}
\label{eq:global}
\ee 
%
The equivalence ratio ($\phi$) is defined as follows:
%
\be
\phi = \frac{(M_{\text{H}_2}/M_{\text{O}_2})_\text{obs}}{(M_{\text{H}_2}/M_{\text{O}_2})_\text{st}}
\label{eq:phi}
\ee
%
The numerator in the RHS represents the fuel-oxygen mass ratio at a given condition
and the denominator represents the stoichiometric ratio of the same quantity. Hence,
$\phi$ = 1 at stoichiometric conditions. The equivalence ratio can be altered
by changing the amount of O$_2$ in the mixture. In the case of a lean
mixture,~\eqref{eq:global} can be written as follows:
%
\be
2\text{H}_2 + \alpha\text{O}_2 \rightarrow 2\text{H}_2\text{O} + (\alpha-1)\text{O}_2 
\hspace{3mm} (\alpha>1)
\label{eq:lean}
\ee 
%
Similarly, for the case when the mixture if fuel rich,~\eqref{eq:global} is modified
as follows:
%
\be
2\text{H}_2 + \alpha\text{O}_2 \rightarrow 2\alpha\text{H}_2\text{O} + 2(1-\alpha)\text{H}_2
\hspace{3mm} (\alpha<1)
\label{eq:rich}
\ee 
%
Eqs.~\eqref{eq:lean} and~\eqref{eq:rich} can be generalized as follows:
%
\be
2\text{H}_2 + \alpha\text{O}_2 \rightarrow 2\min(1,\alpha)\text{H}_2\text{O} + 
\max(\alpha-1,0)\text{O}_2 + \max(0,2-2\alpha)\text{H}_2
\label{eq:gen}
\ee 
%
From the above set of chemical equations, the relationship between $\phi$
and $\alpha$ can be easily obtained as $\left(\phi~=~\frac{1}{\alpha}\right)$.
Since $\phi>1$ corresponds to a rich mixture, and $\phi<1$ corresponds to a
lean mixture, we consider $\phi$ = 2.0 and 0.5 to investigate the two scenarios
respectively. 




