\section{Motivating Examples}
\label{sec:examples}

In Section~\ref{sec:method}, we presented a methodology for screening model
parameters using an approximate estimate of the upper bound on the Sobol'
total effect sensitivity index. The screening process helped identify parameters
that are relatively insignificant with respect to the observable. These insignificant
parameters could thus be fixed at their nominal values thereby reducing the
dimensionality of the problem. 

In this section, we motivate this approach based
on parameter screening to construct low-dimensional PC surrogates by 
applying it to simple test problems: (1) Borehole function, (2) Non-linear Oscillator, and
(3) Semi-linear Elliptic PDE. Since model evaluations in all these test
problems are not expensive, it allows us to compare relative importance of
model parameters,
obtained using the derivative-based sensitivity approach with that obtained
using converged estimates of Sobol' total effect sensitivity index, $\mathcal{T}(\theta_i)$~\cite{Sobol:2001}:

\be
\mathcal{T}(\theta_i) = 
\frac{\mathbb{E}_{\bm{\theta}\sim i}[\mathbb{V}_{\theta_i}(\mathcal{G}|\bm{\theta}_{\sim i})]}{\mathbb{V}(\mathcal{G})}
\ee

\noindent $\mathcal{T}(\theta_i)$ is essentially the ratio of the contribution to the variance of an
observable $\mathcal{G}$ by the parameter $\theta_i$ to the total variance of $\mathcal{G}$.
Additionally, we compare convergence of the PCE constructed in
the full parameter space with those constructed in the reduced space. To assess convergence, 
we rely on the 
leave-one-out-cross-validation error ($\epsilon_{\mbox{\tiny LOO}}$), defined
as follows~\cite{Blatman:2010}:

\be
\epsilon_{\mbox{\tiny LOO}} = 
\frac{\sum_{i=1}^N\left(\mathcal{Y}^{\mbox{\tiny M}}(\bm{\theta}_i) - 
 \mathcal{Y}^{\mbox{\tiny {PC}\textbackslash i}}(\bm{\theta}_i)\right)^2}
{\sum_{i=1}^N
\left(\mathcal{Y}^{\mbox{\tiny M}}(\bm{\theta}_i) - \widetilde{\mu}\right)^2}
\label{eq:loo}
\ee

\noindent where $\widetilde{\mu}~=~\frac{1}{N}\sum_{i=1}^N \mathcal{Y}^{\mbox{\tiny M}}(\bm{\theta}_i)$
is the sample mean of the model response, and $ \mathcal{Y}^{\mbox{\tiny {PC}\textbackslash i}}$
is the PCE surrogate constructed using all but the $i^{\mbox{\tiny{th}}}$ model realization. 
From Eq.~\ref{eq:loo}, it appears that $N$ PCE's are needed to evaluate $\epsilon_{\mbox{\tiny LOO}}$.
However, in practice a modified formulation for $\epsilon_{\mbox{\tiny LOO}}$~\cite{Blatman:2009},
independent of $\mathcal{Y}^{\mbox{\tiny {PC}\textbackslash i}}$ is used.

\subsection{Borehole function}

The borehole function is a benchmark reference problem in sensitivity analysis. It models the discharge
of water ($\mathcal{Q}$) through a borehole in terms of geometrical and physical parameters:

\be
\mathcal{Q} = \frac{2\pi T_u(H_u - H_l)}{\ln\left(\frac{r}{r_w}\right)\left(1 +
\frac{2LT_u}{\ln\left(\frac{r}{r_w}\right)r_w^2K_w} + \frac{T_u}{T_l}\right)}
\label{eq:bore}
\ee

\noindent The radius of influence, $r$ is fixed at 3698.30 m whereas all other parameters
in the RHS of Eq.~\ref{eq:bore} are considered as uncertain. Table~\ref{tab:bore} provides
marginal distributions of the uncertain parameters. 

\begin{table}[htbp]
\renewcommand*{\arraystretch}{1.2}
\begin{center}
\begin{tabular}{|c|c|}
\hline
Parameter & Distribution \\ \hline \hline
$r_w$ (Borehole radius, m) & $\mathcal{N}$(0.1,0.016) \\
$L$ (Borehole length, m) & $\mathcal{U}$[1120,1680] \\
$T_u$ (Transmissivity of upper aquifer, m$^2$/yr) & $\mathcal{U}$[63070,115600] \\
$H_u$ (Potentiometric head of upper aquifer, m) & $\mathcal{U}$[990,1110] \\
$T_l$ (Transmissivity of lower aquifer, m$^2$/yr) & $\mathcal{U}$[63.1,116] \\
$H_l$ (Potentiometric head of lower aquifer, m) & $\mathcal{U}$[700,820] \\
$K_w$ (Borehole hydraulic conductivity, m/yr) & $\mathcal{U}$[9855,12045] \\
\hline
\end{tabular}
\end{center}

\caption{Description and marginal distributions of uncertain parameters in the borehole function
given by Eq.~\ref{eq:bore}.}
\label{tab:bore}
\end{table}

As mentioned earlier in this section, cheap function evaluations of the discharge using 
Eq.~\ref{eq:bore} enable us to compute converged estimates of $\mathcal{T}(\theta_i)$. Plotted in
Figure~\ref{fig:sense_bore} are estimates of these indices corresponding to the uncertain
parameters in the borehole function.




