\section{Motivating Examples}
\label{sec:examples}

In section~\ref{sec:method}, we presented a methodology for screening model
parameters using an approximate estimate of the upper bound on the Sobol'
total effect sensitivity index. The screening process helped identify parameters
that are relatively insignificant with respect to the observable. These insignificant
parameters could thus be fixed at their nominal values thereby reducing the
dimensionality of the problem. 

In this section, we motivate this approach based
on parameter screening to construct low-dimensional PC surrogates by 
applying it to simple test problems: (1) Borehole function, (2) Non-linear Oscillator, and
(3) Semi-linear Elliptic PDE. Since model evaluations in all these test
problems are not expensive, it allows us to compare relative importance of
model parameters,
obtained using the derivative-based sensitivity approach with that obtained
using converged estimates of Sobol' total effect sensitivity index, $\mathcal{T}(\theta_i)$~\cite{Sobol:2001}:

\be
\mathcal{T}(\theta_i) = 
\frac{\mathbb{E}_{\bm{\theta}\sim i}[\mathbb{V}_{\theta_i}(\mathcal{G}|\bm{\theta}_{\sim i})]}{\mathbb{V}(\mathcal{G})}
\ee

\noindent $\mathcal{T}(\theta_i)$ is essentially the ratio of the contribution to the variance of an
observable $\mathcal{G}$ by the parameter $\theta_i$ to the total variance of $\mathcal{G}$.
Additionally, we compare convergence of the PCE constructed in
the full parameter space with those constructed in the reduced space. To assess convergence, 
we rely on the 
leave-one-out-cross-validation error ($\epsilon_{\mbox{\tiny LOO}}$), defined
as follows~\cite{Blatman:2010}:

\be
\epsilon_{\mbox{\tiny LOO}} = 
\frac{\sum_{i=1}^N\left(\mathcal{Y}^{\mbox{\tiny M}}(\bm{\theta}_i) - 
 \mathcal{Y}^{\mbox{\tiny {PC}\textbackslash i}}(\bm{\xi(\theta}_i))\right)^2}
{\sum_{i=1}^N
\left(\mathcal{Y}^{\mbox{\tiny M}}(\bm{\theta}_i) - \widetilde{\mu}\right)^2}
\label{eq:loo}
\ee

\noindent where $\widetilde{\mu}~=~\frac{1}{N}\sum_{i=1}^N \mathcal{Y}^{\mbox{\tiny M}}(\bm{\theta}_i)$
is the sample mean of the model response, and $ \mathcal{Y}^{\mbox{\tiny {PC}\textbackslash i}}$
is the PCE surrogate constructed using all but the $i^{\mbox{\tiny{th}}}$ model realization. 
From Eq.~\ref{eq:loo}, it appears that $N$ PCE's are needed to evaluate $\epsilon_{\mbox{\tiny LOO}}$.
However, in practice a modified formulation for $\epsilon_{\mbox{\tiny LOO}}$~\cite{Blatman:2009},
independent of $\mathcal{Y}^{\mbox{\tiny {PC}\textbackslash i}}$ is used.

The converged low-dimensional PCE is assessed
for predictive accuracy. While several techniques could be used, we consider predictive verification
in an L-2 sense as well as a probabilistic sense. In the former case, we estimate a relative L-2 norm
of the error ($\epsilon_{\mbox{\tiny{L-2}}}$) as follows:

\be
\epsilon_{\mbox{\tiny{L-2}}} = 
\frac{\left[\sum_{i=1}^N\left(\mathcal{Y}^{\mbox{\tiny M}}(\bm{\theta}_i) - 
 \mathcal{Y}^{\mbox{\tiny {PC}}}(\bm{\xi(\theta}_i))\right)^2\right]^{\frac{1}{2}}}
{\left[\sum_{i=1}^N
\left(\mathcal{Y}^{\mbox{\tiny M}}(\bm{\theta}_i)\right)^2\right]^{\frac{1}{2}}}
\label{eq:l2}
\ee

\noindent As discussed earlier in section~\ref{sec:method}, computational cost associated with
L-2 verification is negligible since function evaluations or model realizations obtained during the
screening step could be re-used to estimate $\epsilon_{\mbox{\tiny{L-2}}}$ using Eq.~\ref{eq:l2}.
For the purpose of probabilistic verification, we compare the probability density function of
the observable obtained using PCE's in the full space and the reduced space in the following
test problems. In situations where model runs are computationally expensive, probabilistic verification might
not be feasible. 

\subsection{Borehole function}

The borehole function is a benchmark reference problem in sensitivity analysis. It models the discharge
of water ($\mathcal{Q}$) through a borehole in terms of geometrical and physical parameters:

\be
\mathcal{Q} = \frac{2\pi T_u(H_u - H_l)}{\ln\left(\frac{r}{r_w}\right)\left(1 +
\frac{2LT_u}{\ln\left(\frac{r}{r_w}\right)r_w^2K_w} + \frac{T_u}{T_l}\right)}
\label{eq:bore}
\ee

\noindent The radius of influence, $r$ is fixed at 3698.30 m whereas all other parameters
in the RHS of Eq.~\ref{eq:bore} are considered as uncertain. Table~\ref{tab:bore} provides
marginal distributions of the uncertain parameters. 

\begin{table}[htbp]
\renewcommand*{\arraystretch}{1.2}
\begin{center}
\begin{tabular}{|c|c|}
\hline
Parameter & Distribution \\ \hline \hline
$r_w$ (Borehole radius, m) & $\mathcal{N}$(0.1,0.016) \\
$L$ (Borehole length, m) & $\mathcal{U}$[1120,1680] \\
$T_u$ (Transmissivity of upper aquifer, m$^2$/yr) & $\mathcal{U}$[63070,115600] \\
$H_u$ (Potentiometric head of upper aquifer, m) & $\mathcal{U}$[990,1110] \\
$T_l$ (Transmissivity of lower aquifer, m$^2$/yr) & $\mathcal{U}$[63.1,116] \\
$H_l$ (Potentiometric head of lower aquifer, m) & $\mathcal{U}$[700,820] \\
$K_w$ (Borehole hydraulic conductivity, m/yr) & $\mathcal{U}$[9855,12045] \\
\hline
\end{tabular}
\end{center}

\caption{Description and marginal distributions of uncertain parameters in the borehole function
given by Eq.~\ref{eq:bore}.}
\label{tab:bore}
\end{table}

As mentioned earlier in this section, cheap function evaluations of the discharge using 
Eq.~\ref{eq:bore} enable us to compute converged estimates of $\mathcal{T}(\theta_i)$. Plotted in
Figure~\ref{fig:sense_bore} are estimates of these indices corresponding to the uncertain
parameters in the borehole function using 10$^6$ pseudo-random
samples\footnote{Although $\mathcal{T}(\theta_i)$ 
might converge with much fewer samples depending upon the model, we consider a large number that
typically ensures a converged estimate for the purpose of illustration.} in the input parameter domain. 
These estimates are used to verify fidelity of
parameter screening based on the methodology presented in Section~\ref{sec:method}. 
In Figure~\ref{fig:screen_bore}, we plot estimates of the screening parameter ($\hat{\mathcal{C}_i\mu_i}$)
for a wide range of the number of samples used for approximating $\mu_i$ using Eq.~\ref{eq:mu}.
Estimates for $\hat{\mathcal{C}_i\mu_i}$ are found to be in excellent agreement with $\mathcal{T}(\theta_i)$
even when small number of samples (5 -- 10) are used. 
Consequently, the relative importance of uncertain 
parameters in the borehole function is found to be consistent with predictions based on the Sobol' index. 
In the considered intervals for the uncertain parameters, it is clear that the discharge is insensitive to
$T_u$ and $T_l$. Moreover, the sensitivity towards $K_w$ is also small. We exploit this result to reduce
the dimensionality of the problem by 
discounting the uncertainties in $T_u$, $T_l$, and $K_w$, while using nominal values for these
parameters instead.  

Our goal as discussed is to gain computational advantage by constructing surrogates in a 
reduced input parameter space. To this end, we construct PCE's in 5D and 4D spaces by
fixing $\{T_u,T_l\}$ in the former and additionally fixing $K_w$ in the latter at their respective
mean values. In Figure~\ref{fig:conv_bore}, we compare convergence of PCE's constructed
in the full space (7D) with those constructed in the two reduced spaces (4D and 5D) using
$\epsilon_{\mbox{\tiny{LOO}}}$ (Eq.~\ref{eq:loo}). As expected, it is observed that the PCE
constructed in the 4D space converges at a much faster rate. For instance, if a PCE with
$\mathcal{O}(10^{-4})$ accuracy is sought, we need function evaluations at only about
50 sample points in the 4D parameter space whereas the number of samples needed in the full 
7D space seems much higher. 

As discussed earlier in this section, the reduced order PCE's are verified for predictive accuracy in 
a least-squares sense and a probabilistic sense. Estimates for $\epsilon_{\mbox{\tiny{L-2}}}$
based on 50 samples and corresponding function evaluations in the original 7D parameter space
were found to be 0.0551 and 0.0112 for the 4D and 5D PCE's respectively. In other words, the 4D
PCE is accurate within 5.52$\%$ and the 5D PCE is accurate within 1.12$\%$ of predictions based on
the borehole function. It is interesting to note that although the order of convergence in the case of a 5D
PCE is lower (Figure~\ref{fig:conv_bore}), its predictive accuracy is higher than the 4D PCE. A possible
explanation for this observation is that above a certain order of convergence, the uncertainty in $k_w$ 
contributes much more towards predictive accuracy of the reduced order PCE. It is therefore critical
to account for the order of PCE convergence as well as its predictive accuracy for a given application,
as highlighted earlier in section~\ref{sec:method}. 

Figure~\ref{fig:pdfcomp_bore} illustrates a comparison of the PDF's of the discharge, $\mathcal{Q}$
obtained by propagating 10$^6$ Monte-Carlo (MC) samples through the 7D PCE in the
original input parameter domain as well as the reduced order PCE's constructed in 4 and 5 dimensions. 
It is evident from this plot, that the PDF's agree quite favorably with each other. Consequently, it can
be said that the reduced order PCE is verified in a probabilistic sense. In other words, the mode as
well as the uncertainty in the observable is reliably captured and predicted by the reduced order PCE. 
 
\subsection{Non-linear Oscillator}

































