\section*{Abstract}
Surrogate modeling has become a critical component of scientific computing
in situations involving expensive model evaluations. However, training a
surrogate model can be remarkably challenging and even computationally
prohibitive in the case of intensive simulations and large-dimensional
systems.
We develop a systematic approach for surrogate model
construction in reduced input parameter spaces.
%using derivative-based global sensitivity measures~(DGSMs)~\cite{Sobol:2010}. 
%
A sparse set of model evaluations in the original input space is used to  
approximate derivative based global sensitivity measures (DGSMs) 
for individual uncertain inputs of the model.
An iterative screening procedure is developed that exploits DGSM estimates in
order to identify the \emph{unimportant} inputs. The screening procedure forms
an integral part of an overall framework for adaptive construction of a
surrogate in the reduced space. The framework is tested for computational
efficiency through an initial implementation in simple test cases such as the
classic Borehole function, and a semilinear elliptic PDE with a random source
term. 
%Motivated by these experiences, the framework is fully implemented to
The framework is then deployed for a realistic application from chemical
kinetics, where we study the ignition delay in an H$_2$/O$_2$ reaction
mechanism with 19 uncertain rate constants.  It is observed that significant
computational gains can be attained by constructing accurate low-dimensional
surrogates using the proposed framework.
\vspace{2mm}

\noindent
\textbf{Keywords}: Global sensitivity analysis,
polynomial chaos, parameter screening, surrogate modeling
 
