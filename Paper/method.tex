\section{Methodology}
\label{sec:method}

%Stage1: QoI selection for accurate estimation of derivatives and PCE, Stage 2: Parameter Screening, Stage 3: Reduced-order Surrogate Verification
%
%1. Goal is to estimate the upper bound on Sobol total-effect index using DGSM. 
%2. QoI should be differentiable w.r.t all the parameters. 
%3. Derivative could be estimated analytically or numerically.
%4. The expected value of the partial derivative w.r.t a given parameter is approximated over a few samples.
%5. Gradually enhance the sample size until some degree of convergence is established.
%6. Note that model runs at the full parameter space will be used for verification of the ROS. However, one could consider
%relaxing the convergence criterion to reduce computational costs. 
%7. Consider the product of Ci and mu_i to screen parameters.i

In this section, we outline the underlying methodology for constructing a 
reduced-order surrogate using sensitivity analysis. The term `reduced-order'
in the present context implies that the surrogate is constructed in a 
subspace that sufficiently captures the uncertainty in the model output. 
In other words, the dimensionality of the polynomial basis functions in
the case of PC surrogates is effectively reduced. A possible approach to 
constructing a reduced-order surrogate involves sensitivity analysis of the
uncertain model parameters with respect to a given output, and thereby
disregard parameters considered as unimportant. For this purpose, global
sensitivity analysis based on Sobol indices could be pursued. However,
as discussed earlier in section~\ref{sec:intro}, determining converged 
estimates of Sobol sensitivity indices typically requires tens of thousands
of model evaluations. Consequently, the exercise becomes prohibitive in
situations where the model runs are compute-intensive.    


 
